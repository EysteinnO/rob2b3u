\section{Inngangur}
Hér skal gera lýsingu á verkefninu þ.e hvað,  hvernig og  hvaða forritunarmál, fyrir hverja og hvaða notagildi verkefnið hefur. Minnst 500 orð. Notagildi skiptir miklumáli, reynið að sjá fyrir ykkur hverjir geti notað vélmennið ykkar og í hvaða tilgangi.  Þá kemur í ljós að 500 orð er frekar lítið :-) Hér er gott að byrja á því að lesa til um Arduino en allt hjá þeim er open-sourse og svo er hægt að lesa sér til um efnið í útgefnum bókum sem "programming Arduino \cite{monk} Skoðið vel heimildaskrá og skránna mybib.bib. Hér er gott að lýsa högun kerfisins með orðum og mynd sem þið getið gert í draw.io sjá mynd: 


Hugmyndin að verkefninu er sú að hafa þjónustu fyrir sumarhús sem býður upp á það að stilla t.d. hita á ofnum, lesa inn gildi af gasskynjurum, hitaskynjurum og hreyfiskynjurum. Forritunarmálið að baki vélmennisins verður Arduino, hann sendir gögn á PHP síðu sem hendir gögnunum inná SQL server. Notað verður Digital Ocean vefhýsingu fyrir PHP síðuna. Vefsíðan sjálf verður svo fyrir notendann að fylgjast með öllum gildum sem vélmennið sækir frá sumarhúsinu. Sett væri upp login kerfi fyrir notendann þar sem hann einn kemst að þessum upplýsingum og getur fylgst með. Einnig væri hægt að setja upp SMS kerfi ef hitinn á húsinu myndi skyndilega detta niður til þess að koma í veg fyrir frosnar lagnir sem hentar einstaklega vel á Íslandi. Hægt væri að útfæra þetta enn stærra og fara upp í stór fyrirtæki eins og hótel eða gistiheimili út á landi. 

\begin{figure}[h]
\includegraphics[scale=.3]{img/system}
\end{figure}